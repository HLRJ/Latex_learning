%! Author = zhenxiang
%! Date = 2023/3/27

\PassOptionsToPackage{quiet}{xeCJK}  % 抑制无意义的警告
% Preamble
\documentclass[11pt]{ctexart}


% Packages
\usepackage{amsmath}
% graphicx
\usepackage{graphicx}
% 页面设置
\usepackage{geometry}
\geometry{left=2.5cm, right=2.5cm, top=2.5cm, bottom=2.5cm}
% codes
\usepackage{listings}
\usepackage{xcolor}
\usepackage{color}
\definecolor{mygreen}{rgb}{0,0.6,0}
\definecolor{mygray}{rgb}{0.5,0.5,0.5}
\definecolor{mymauve}{rgb}{0.58,0,0.82}
\lstset{ %
	backgroundcolor=\color{white},   % choose the background color; you must add \usepackage{color} or \usepackage{xcolor}
	basicstyle=\ttfamily,            % the size of the fonts that are used for the code
	breakatwhitespace=false,         % sets if automatic breaks should only happen at whitespace
	breaklines=true,                 % sets automatic line breaking
	captionpos=b,                    % sets the caption-position to bottom
	commentstyle=\ttfamily\color{mygreen},
	% comment style
	deletekeywords={},               % if you want to delete keywords from the given language
	escapeinside={},                 % if you want to add LaTeX within your code
	extendedchars=true,              % lets you use non-ASCII characters; for 8-bits encodings only, does not work with UTF-8
	frame=single,                    % adds a frame around the code
	keepspaces=true,                 % keeps spaces in text, useful for keeping indentation of code (possibly needs columns=flexible)
	keywordstyle=\color{blue},       % keyword style
	language=bash,                    % the language of the code
	morekeywords={},                 % if you want to add more keywords to the set
	numbers=left,                    % where to put the line-numbers; possible values are (none, left, right)
	numbersep=5pt,                   % how far the line-numbers are from the code
	numberstyle=\tiny\color{mygray}, % the style that is used for the line-numbers
	rulecolor=\color{black},         % if not set, the frame-color may be changed on line-breaks within not-black text (e.g. comments (green here))
	showspaces=false,                % show spaces everywhere adding particular underscores; it overrides 'showstringspaces'
	showstringspaces=false,          % underline spaces within strings only
	showtabs=false,                  % show tabs within strings adding particular underscores
	stepnumber=1,                    % the step between two line-numbers. If it's 1, each line will be numbered
	stringstyle=\color{mymauve},     % string literal style
	tabsize=2,                       % sets default tabsize to 2 spaces
	title=\lstname                   % show the filename of files included with \lstinputlisting; also try caption instead of title
}
%url
\usepackage{hyperref}
% ref
\usepackage{gbt7714}
\bibliographystyle{gbt7714-numerical}
\begin{document}
	
\section{You will get}

\begin{itemize}
	\item 环境配置、编辑器使用
	\item 简单操作
	\item 套用模板
	\item  遇到问题怎么解决
\end{itemize}

\newpage

\section{Install}

\subsection{安装介绍}

安装步骤,参考\href{https://github.com/OsbertWang/install-latex-guide-zh-cn/releases}{guide手册}

从\href{https://mirrors.tuna.tsinghua.edu.cn/CTAN/systems/texlive/Images/}{清华下载镜像源 texlive2023.iso}

\begin{figure}[ht]
	\centering
	\includegraphics[width=1.0\textwidth]{images/texlive2023Images.png}
	\caption{镜像源~ 选择texlive2023.iso}
	\label{fig:2023image}
\end{figure}

安装过程

\begin{figure}[ht]
	\centering
	\includegraphics[width=0.4\textwidth]{images/installing_1.png}
	\qquad
	\includegraphics[width=0.4\textwidth]{images/installing_2.png}
	\caption{安装过程}
	\label{fig:install}
\end{figure}



\newpage

\begin{figure}[ht]
	\centering
	\includegraphics[width=0.5\textwidth]{images/texstudio.png}
	\caption{下载你的平台版本}
	\label{fig:texstudio}
\end{figure}

编辑器下载\href{https://www.texstudio.org/}{texstudio}



ubuntu下安装texstudio

\begin{lstlisting}
	sudo add-apt-repository ppa:sunderme/texstudio
	sudo apt update
	sudo apt install texstudio
\end{lstlisting}

以及其他编辑器 如texmaker 、jetbrains ( pycharm、clion ) 装插件、配置vscode


配置texlive2023环境变量

\begin{lstlisting}
	vi ~/.bashrc
	export PATH=/usr/local/texlive/2023/bin/x86_64-linux:$PATH
	export MANPATH=/usr/local/texlive/2023/texmf-dist/doc/man:$MANPATH
	export INFOPATH=/usr/local/texlive/2023/texmf-dist/doc/info:$INFOPATH
\end{lstlisting}




\newpage

\section{use}

\subsection{local}

\subsubsection{ 插入代码}

\subsubsection{插入公式}

\subsubsection{图片}


 
 \subsubsection{表格}
 
 \subsubsection{字体设置}
 
 \subsubsection{页面设置}
 
 \subsubsection{目录}
 
 \subsubsection{文件结构}
 
 \subsubsection{插入链接}
 
 \subsubsection{自定义命令}
 
 \subsubsection{引用文献}
 
 \subsection{online}
 
 在线的编辑器 \href{www.overleaf.com}{国际overleaf}
 
 \href{cn.overleaf.com}{国内版本overleaf}
 
 国产的\href{www.texpage.com}{texpage}
 
 
 \subsection{for me personally}
 
 pycharm latex模板配置, pycharm实现git简单操作push到自己私有仓库,Texstudio编写tex文件
 
 \section{template}

使用模板, 以nips imcl为例\cite{kaelbling1996reinforcement}


\section{solve problem}

 chatgpt,查阅文档  texdoc name\cite{lecun2015deep}

参考的文档,一份简单的介绍\cite{de2019causal}

arxiv上下载latex版本文章\cite{wen2020fighting}

install







ishort-zh-cn

https://github.com/CTeX-org/lshort-zh-cn/releases

https://github.com/xinychen/latex-cookbook

https://github.com/luong-komorebi/Begin-Latex-in-minutes

https://github.com/dspinellis/latex-advice

https://tug.org/

https://ctan.org/

https://texdoc.org/index.html

https://ctex.org/

https://tex.stackexchange.com/

https://www.mathcha.io/

https://www.latexlive.com/home\#\#


\bibliography{bib/ref_learn.bib}


\end{document}